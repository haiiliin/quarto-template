% Options for packages loaded elsewhere
\PassOptionsToPackage{unicode}{hyperref}
\PassOptionsToPackage{hyphens}{url}
\PassOptionsToPackage{dvipsnames,svgnames,x11names}{xcolor}
%
\documentclass[
  authoryear,
  review,
  3p,
  onecolumn]{elsarticle}

\usepackage{amsmath,amssymb}
\usepackage{iftex}
\ifPDFTeX
  \usepackage[T1]{fontenc}
  \usepackage[utf8]{inputenc}
  \usepackage{textcomp} % provide euro and other symbols
\else % if luatex or xetex
  \usepackage{unicode-math}
  \defaultfontfeatures{Scale=MatchLowercase}
  \defaultfontfeatures[\rmfamily]{Ligatures=TeX,Scale=1}
\fi
\usepackage{lmodern}
\ifPDFTeX\else  
    % xetex/luatex font selection
\fi
% Use upquote if available, for straight quotes in verbatim environments
\IfFileExists{upquote.sty}{\usepackage{upquote}}{}
\IfFileExists{microtype.sty}{% use microtype if available
  \usepackage[]{microtype}
  \UseMicrotypeSet[protrusion]{basicmath} % disable protrusion for tt fonts
}{}
\makeatletter
\@ifundefined{KOMAClassName}{% if non-KOMA class
  \IfFileExists{parskip.sty}{%
    \usepackage{parskip}
  }{% else
    \setlength{\parindent}{0pt}
    \setlength{\parskip}{6pt plus 2pt minus 1pt}}
}{% if KOMA class
  \KOMAoptions{parskip=half}}
\makeatother
\usepackage{xcolor}
\setlength{\emergencystretch}{3em} % prevent overfull lines
\setcounter{secnumdepth}{5}
% Make \paragraph and \subparagraph free-standing
\makeatletter
\ifx\paragraph\undefined\else
  \let\oldparagraph\paragraph
  \renewcommand{\paragraph}{
    \@ifstar
      \xxxParagraphStar
      \xxxParagraphNoStar
  }
  \newcommand{\xxxParagraphStar}[1]{\oldparagraph*{#1}\mbox{}}
  \newcommand{\xxxParagraphNoStar}[1]{\oldparagraph{#1}\mbox{}}
\fi
\ifx\subparagraph\undefined\else
  \let\oldsubparagraph\subparagraph
  \renewcommand{\subparagraph}{
    \@ifstar
      \xxxSubParagraphStar
      \xxxSubParagraphNoStar
  }
  \newcommand{\xxxSubParagraphStar}[1]{\oldsubparagraph*{#1}\mbox{}}
  \newcommand{\xxxSubParagraphNoStar}[1]{\oldsubparagraph{#1}\mbox{}}
\fi
\makeatother

\usepackage{color}
\usepackage{fancyvrb}
\newcommand{\VerbBar}{|}
\newcommand{\VERB}{\Verb[commandchars=\\\{\}]}
\DefineVerbatimEnvironment{Highlighting}{Verbatim}{commandchars=\\\{\}}
% Add ',fontsize=\small' for more characters per line
\usepackage{framed}
\definecolor{shadecolor}{RGB}{241,243,245}
\newenvironment{Shaded}{\begin{snugshade}}{\end{snugshade}}
\newcommand{\AlertTok}[1]{\textcolor[rgb]{0.68,0.00,0.00}{#1}}
\newcommand{\AnnotationTok}[1]{\textcolor[rgb]{0.37,0.37,0.37}{#1}}
\newcommand{\AttributeTok}[1]{\textcolor[rgb]{0.40,0.45,0.13}{#1}}
\newcommand{\BaseNTok}[1]{\textcolor[rgb]{0.68,0.00,0.00}{#1}}
\newcommand{\BuiltInTok}[1]{\textcolor[rgb]{0.00,0.23,0.31}{#1}}
\newcommand{\CharTok}[1]{\textcolor[rgb]{0.13,0.47,0.30}{#1}}
\newcommand{\CommentTok}[1]{\textcolor[rgb]{0.37,0.37,0.37}{#1}}
\newcommand{\CommentVarTok}[1]{\textcolor[rgb]{0.37,0.37,0.37}{\textit{#1}}}
\newcommand{\ConstantTok}[1]{\textcolor[rgb]{0.56,0.35,0.01}{#1}}
\newcommand{\ControlFlowTok}[1]{\textcolor[rgb]{0.00,0.23,0.31}{\textbf{#1}}}
\newcommand{\DataTypeTok}[1]{\textcolor[rgb]{0.68,0.00,0.00}{#1}}
\newcommand{\DecValTok}[1]{\textcolor[rgb]{0.68,0.00,0.00}{#1}}
\newcommand{\DocumentationTok}[1]{\textcolor[rgb]{0.37,0.37,0.37}{\textit{#1}}}
\newcommand{\ErrorTok}[1]{\textcolor[rgb]{0.68,0.00,0.00}{#1}}
\newcommand{\ExtensionTok}[1]{\textcolor[rgb]{0.00,0.23,0.31}{#1}}
\newcommand{\FloatTok}[1]{\textcolor[rgb]{0.68,0.00,0.00}{#1}}
\newcommand{\FunctionTok}[1]{\textcolor[rgb]{0.28,0.35,0.67}{#1}}
\newcommand{\ImportTok}[1]{\textcolor[rgb]{0.00,0.46,0.62}{#1}}
\newcommand{\InformationTok}[1]{\textcolor[rgb]{0.37,0.37,0.37}{#1}}
\newcommand{\KeywordTok}[1]{\textcolor[rgb]{0.00,0.23,0.31}{\textbf{#1}}}
\newcommand{\NormalTok}[1]{\textcolor[rgb]{0.00,0.23,0.31}{#1}}
\newcommand{\OperatorTok}[1]{\textcolor[rgb]{0.37,0.37,0.37}{#1}}
\newcommand{\OtherTok}[1]{\textcolor[rgb]{0.00,0.23,0.31}{#1}}
\newcommand{\PreprocessorTok}[1]{\textcolor[rgb]{0.68,0.00,0.00}{#1}}
\newcommand{\RegionMarkerTok}[1]{\textcolor[rgb]{0.00,0.23,0.31}{#1}}
\newcommand{\SpecialCharTok}[1]{\textcolor[rgb]{0.37,0.37,0.37}{#1}}
\newcommand{\SpecialStringTok}[1]{\textcolor[rgb]{0.13,0.47,0.30}{#1}}
\newcommand{\StringTok}[1]{\textcolor[rgb]{0.13,0.47,0.30}{#1}}
\newcommand{\VariableTok}[1]{\textcolor[rgb]{0.07,0.07,0.07}{#1}}
\newcommand{\VerbatimStringTok}[1]{\textcolor[rgb]{0.13,0.47,0.30}{#1}}
\newcommand{\WarningTok}[1]{\textcolor[rgb]{0.37,0.37,0.37}{\textit{#1}}}

\providecommand{\tightlist}{%
  \setlength{\itemsep}{0pt}\setlength{\parskip}{0pt}}\usepackage{longtable,booktabs,array}
\usepackage{calc} % for calculating minipage widths
% Correct order of tables after \paragraph or \subparagraph
\usepackage{etoolbox}
\makeatletter
\patchcmd\longtable{\par}{\if@noskipsec\mbox{}\fi\par}{}{}
\makeatother
% Allow footnotes in longtable head/foot
\IfFileExists{footnotehyper.sty}{\usepackage{footnotehyper}}{\usepackage{footnote}}
\makesavenoteenv{longtable}
\usepackage{graphicx}
\makeatletter
\newsavebox\pandoc@box
\newcommand*\pandocbounded[1]{% scales image to fit in text height/width
  \sbox\pandoc@box{#1}%
  \Gscale@div\@tempa{\textheight}{\dimexpr\ht\pandoc@box+\dp\pandoc@box\relax}%
  \Gscale@div\@tempb{\linewidth}{\wd\pandoc@box}%
  \ifdim\@tempb\p@<\@tempa\p@\let\@tempa\@tempb\fi% select the smaller of both
  \ifdim\@tempa\p@<\p@\scalebox{\@tempa}{\usebox\pandoc@box}%
  \else\usebox{\pandoc@box}%
  \fi%
}
% Set default figure placement to htbp
\def\fps@figure{htbp}
\makeatother

\usepackage{ctex}
\usepackage{lineno}
\makeatletter
\@ifpackageloaded{caption}{}{\usepackage{caption}}
\AtBeginDocument{%
\ifdefined\contentsname
  \renewcommand*\contentsname{Table of contents}
\else
  \newcommand\contentsname{Table of contents}
\fi
\ifdefined\listfigurename
  \renewcommand*\listfigurename{List of Figures}
\else
  \newcommand\listfigurename{List of Figures}
\fi
\ifdefined\listtablename
  \renewcommand*\listtablename{List of Tables}
\else
  \newcommand\listtablename{List of Tables}
\fi
\ifdefined\figurename
  \renewcommand*\figurename{Figure}
\else
  \newcommand\figurename{Figure}
\fi
\ifdefined\tablename
  \renewcommand*\tablename{Table}
\else
  \newcommand\tablename{Table}
\fi
}
\@ifpackageloaded{float}{}{\usepackage{float}}
\floatstyle{ruled}
\@ifundefined{c@chapter}{\newfloat{codelisting}{h}{lop}}{\newfloat{codelisting}{h}{lop}[chapter]}
\floatname{codelisting}{Listing}
\newcommand*\listoflistings{\listof{codelisting}{List of Listings}}
\makeatother
\makeatletter
\makeatother
\makeatletter
\@ifpackageloaded{caption}{}{\usepackage{caption}}
\@ifpackageloaded{subcaption}{}{\usepackage{subcaption}}
\makeatother
\journal{Computers and Geotechnics}

\usepackage[]{natbib}
\bibliographystyle{elsarticle-harv}
\usepackage{bookmark}

\IfFileExists{xurl.sty}{\usepackage{xurl}}{} % add URL line breaks if available
\urlstyle{same} % disable monospaced font for URLs
\hypersetup{
  pdftitle={Quarto template for writing academic papers},
  pdfauthor={Hai-Lin Wang},
  pdfkeywords={Manuscript, Template},
  colorlinks=true,
  linkcolor={blue},
  filecolor={Maroon},
  citecolor={Blue},
  urlcolor={Blue},
  pdfcreator={LaTeX via pandoc}}


\setlength{\parindent}{6pt}
\begin{document}

\begin{frontmatter}
\title{Quarto template for writing academic papers}
\author[1,2]{Hai-Lin Wang%
\corref{cor1}%
}
 \ead{hailin.wang@connect.polyu.hk} 

\affiliation[1]{organization={Department of Civil and Environmental
Engineering, The Hong Kong Polytechnic University, Hong Kong,
China},,postcodesep={}}
\affiliation[2]{organization={Department of Geotechnical Engineering,
College of Civil Engineering, Tongji University, Shanghai,
China},,postcodesep={}}

\cortext[cor1]{Corresponding author}

        
\begin{abstract}
The \href{https://github.com/haiiliin/quarto-template}{quarto-template}
is a Quarto template for writing academic papers. The template is based
on the \href{https://quarto.org}{Quarto} document system, which is a
document system that supports the entire research lifecycle, from
initial exploration to final publication. The template provides a simple
and clean layout for writing academic papers, which is suitable for
researchers who want to focus on the content of the paper rather than
the formatting.
\end{abstract}





\begin{keyword}
    Manuscript \sep 
    Template
\end{keyword}
\end{frontmatter}
    
\linenumbers


\section{Features}\label{features}

\begin{itemize}
\tightlist
\item
  Write academic papers in markdown, focus on the content rather than
  the formatting. See \href{https://quarto.org/docs/guide/}{Quarto
  documentation} for more information.
\item
  Export to html, docx, and pdf formats using the Quarto document system
  (\texttt{make\ paper}).
\item
  Make releases with markdown, tex, docx, and pdf documents archived
  (\texttt{make\ release-\textless{}tag\textgreater{}}). Generate a diff
  file compared with a previous release using latexdiff
  (\texttt{make\ diff\ previous=\textless{}previous-tag\textgreater{}\ current=\textless{}current-tag\textgreater{}}).
\item
  Publish the HTML version of the paper to the web using GitHub Pages.
\end{itemize}

\section{Prerequisites}\label{prerequisites}

To use the template, you need to have the Quarto document system
installed on your computer. You can install Quarto by following the
instructions on the
\href{https://quarto.org/docs/getting-started/}{Quarto website}.

To render the paper to PDF, you also need to have a LaTeX distribution
installed on your computer. You can install LaTeX by following the
instructions on the \href{https://www.latex-project.org/get/}{LaTeX
website}. You can also use the
\href{https://yihui.org/tinytex/}{tinytex} package for a lightweight
LaTeX distribution:

\begin{Shaded}
\begin{Highlighting}[]
\ExtensionTok{quarto}\NormalTok{ install tinytex}
\end{Highlighting}
\end{Shaded}

See \href{https://quarto.org/docs/output-formats/pdf-engine.html}{PDF
Engines in Quarto} for more information.

The GNU Make utility is required to run the commands in the
\texttt{Makefile}. You can install GNU Make by following the
instructions on the \href{https://www.gnu.org/software/make/}{GNU Make
website}. If you are using Windows, you can install GNU Make from the
\href{http://gnuwin32.sourceforge.net/packages/make.htm}{GnuWin32}
project.

To clone the repository and run the commands in the \texttt{Makefile},
you need to have Git installed on your computer. You can install Git by
following the instructions on the \href{https://git-scm.com/}{Git
website}. If you prefer to use a graphical user interface for Git, you
can install \href{https://desktop.github.com/}{GitHub Desktop}.

\section{Usages}\label{usages}

\subsection{Setup the project}\label{setup-the-project}

The template is designed to be easy to use, with minimal configuration
required. To use the template, simply fork the
\href{https://github.com/haiiliin/quarto-template}{quarto-template}
repository. You can then clone the repository to your local computer
using Git:

\begin{Shaded}
\begin{Highlighting}[]
\FunctionTok{git}\NormalTok{ clone https://github.com/}\OperatorTok{\textless{}}\NormalTok{your{-}username}\OperatorTok{\textgreater{}}\NormalTok{/}\OperatorTok{\textless{}}\NormalTok{your{-}repository}\OperatorTok{\textgreater{}}\NormalTok{.git}
\end{Highlighting}
\end{Shaded}

Or you can simply open the repository in GitHub Desktop from the
browser.

\subsection{Start writing}\label{start-writing}

You can now start writing your paper in the \texttt{paper.md} file. For
more information on how to write papers using the Quarto document
system, please refer to the \href{https://quarto.org/docs/guide/}{Quarto
documentation}.

To configure the metadata of the paper, such as the title, authors,
abstract, and article template to use, you can edit the YAML front
matter at the beginning of the \texttt{paper.md} file and the
\texttt{\_quarto.yml} file.

\subsection{Render the paper}\label{render-the-paper}

To render the paper to PDF, run the following commands in the terminal:

\begin{Shaded}
\begin{Highlighting}[]
\FunctionTok{make}\NormalTok{ deps   }\CommentTok{\# Fetch the extensions for journal articles}
\FunctionTok{make}\NormalTok{ paper  }\CommentTok{\# Render the paper to docx, html, and pdf files}
\FunctionTok{make}\NormalTok{ clean  }\CommentTok{\# Clean up the intermediate files}
\end{Highlighting}
\end{Shaded}

\subsection{Making releases}\label{making-releases}

You can use the following to make a release, markdown, tex, docx, and
pdf documents will be archived in the directory
\texttt{releases/\textless{}tag\textgreater{}}:

\begin{Shaded}
\begin{Highlighting}[]
\NormalTok{make release{-}\textless{}tag\textgreater{}}
\end{Highlighting}
\end{Shaded}

You can also use the following command to generate a diff file compared
with a previous release using latexdiff:

\begin{Shaded}
\begin{Highlighting}[]
\NormalTok{make diff previous=\textless{}previous{-}tag\textgreater{} current=\textless{}current{-}tag\textgreater{}}
\end{Highlighting}
\end{Shaded}

\subsection{Commit and push}\label{commit-and-push}

After writing the paper, commit and push the changes to your repository.
You can then share the link to the repository with your collaborators or
submit the paper to a journal for publication.

A html version of the paper will be published to the \texttt{gh-pages}
branch of the repository after every commit. You can turn on the GitHub
Pages feature in the repository settings to publish the html version of
the paper to the web. You can then view the paper online by visiting the
link provided in the repository settings.

\begin{quote}
{[}!Note{]} You need to turn on the \textbf{Read and write permissions}
for the \textbf{Actions} in the \textbf{Settings} of your repository to
grant the permission for the GitHub Actions to upload the rendered
files.
\end{quote}


  \bibliography{references.bib}



\end{document}
